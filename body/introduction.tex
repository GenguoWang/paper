\chapter{绪论}
\label{chap:introduction}

本章介绍了本课题的研究背景和研究意义。并简单的阐述了本课题使用的研究方法。

\section{研究背景和意义}
随着计算机硬件和网络的不断发展,计算机的使用方式也慢慢的发生了变化,之前的单机使用习惯已经越来越淡化,而对于网络则变得越来越依赖。人们从网络上获取最新的信息咨询,与他人随时随地交流,快速便捷地购物。随着网络的服务的百花齐放,新的计算方式云计算也随之蓬勃发展。

云计算\cite{CloudComputing}是通过网络进行相关服务的部署,使得用户可以远程的在线访问这些服务,例如数据存储,弹性计算。云计算提供商通常使用计算机集群来提供这些服务,并通过虚拟化的方式进行多个用户间资源的共享。与租用硬件不同,云计算按照资源的使用进行计价,这样用户无需担心自己计算需求的变化导致浪费硬件资源,而云服务商通过资源共享,可以最大化的榨取硬件资源的利用价值。云计算使得计算能力变得像电,燃气一样可以流通,用户只需按需购买。

云计算有着大规模、虚拟化、高可靠性、通用性、高扩展性、廉价的特点。得益于这些特点,越来越多的个人和企业选择把数据存放到的云存储上,把自己的业务部署到云计算上。目前各大厂商比如亚马逊,微软,谷歌,阿里云都给用户提供了云计算服务。

云计算其中一种的服务方式是给用户提供数据的存储服务。用户可以把自己的数据放到云平台上,然后只要能在能够访问互联网的环境下,用户就可以随时随地使用任意设备访问自己的数据,方便的进行读写,同步,更新等操作。出了读写这些操作,能够对云存储服务上的数据进行搜索也是用户急切需要的功能。

但和本地环境不同,云计算有着潜在的危险性\cite{CloudComputing}。在享受云计算带来的便利时,我们怎么保证我们得到的是我们真正想要的。比如对于云上的搜索任务,我们怎么保证它真的是按照我们的要求完成任务的呢?目前各大厂商都没有提供一种可以让用户进行验证的机制。云搜索可能会由于服务商的硬件错误,算法错误,或者是受到黑客攻击导致搜索结果是错误的。而没有验证机制,用户可能在毫无察觉的情况下使用了错误的搜索结果而蒙受损失。而且云搜索需要耗费服务商的大量计算资源,服务商为了节省资源可能恶意的只进行部分计算,只返回不完整的结果给用户。这让用户的利益更加无法得到保障。

这些风险的存在,导致很多对安全性正确性比较在意的用户只能放弃了云存储,使得他们无法享用云计算带来的那些好处。而服务商也因此失去了不少用户资源。
针对这些情况,如果能找到一种低成本的验证方式,那么就可以很好的解决了这个问题。设想有一种方式,服务商付出完全可以承受额外计算和存储压力就可以为每一次搜索计算一个证明返回给用户,同时用户也可以快速有效地进行正确性验算,不至于影响到用户体验。这对于云计算服务商以及用户都是非常有意义的。

\section{国内外研究现状}
本节我们介绍下关于文档搜索与结果验证方面的国内外研究现状。

\subsection{文档搜索方法和结果验证的关系}
		云存储上的文档搜索和本地的文档搜索没有什么本质的区别。这一块已经积累了很多研究以及有很多成熟的产品。对于不同的搜索方法,验证结果的正确的方法也有很大的差异。所以我们打算只选择一种最普遍的全文搜索的方法来进行我们的研究:先对文档进行预处理,然后建立反向索引。通过反向索引进行多关键词的搜索。
\subsection{搜索结果验证}
				 验证或者保证第三方计算的正确性,并不是一个新的方向。在早期,交互式证明系统就被提出来用于证明第三方计算的正确性。交互式证明系统分为一个假设计算力无限的证明者和计算力有限的验证者。通过类似提问回答的交互方式,证明者试图证明它给出的结论是正确的,而验证者则要判断证明者的结论是否是可信的。Arthur–Merlin协议实现了一种概率上宽容了的交互性验证系统。还有一类是概率验证模型,它通过随机的检测一些部分的结论,来一定概率上确定该结论是否是正确的。最近的在这两类上一些研究在时间空间复杂度上有着极大的提高。但是生成证明还是比计算出结果要耗费上一个数量级以上的时间,这极大的影响了它们的可用性。除了验证计算结果这一方式外,也有研究提供一些计算结果保证正确的机制。比如在执行前验证可执行文件的完整性,提供第三方监控层监控执行过程。但这无法保证云计算提供商或者破坏分子不去攻击内存中的数据,或者更改执行结果。具体到验证云计算搜索结果正确性验证,有研究提供了一种使用不可逆哈希强迫云计算提供商要进行正确的搜索计算才能给出哈希验算结果。该方法有不错的时间效率,但是也有很多局限性,比如不能支持单个搜索。
\subsection{数据安全}
						云存储依赖于第三方,难免会涉及敏感数据,数据安全是云计算研究中的关键问题。有研究在不安全存储上建立安全文件系统。为了保证数据的隐私性,可以采用对称加密方法(比如,AES)对数据加密之后进行上传,这样就可以保证除了密钥拥有者,别人无法看到数据的明文。

						    涉及到搜索,我们可能要需要把索引文件上传到服务端。为了保证索引的安全性,有研究给出,可以使用对称加密给索引的每个关键词都加密,这样就可以防止信息泄露。对于词频信息,有研究指出可以将不同关键词进行组合来进行伪造隐藏。

\section{研究内容}
本课题分析目前现有的搜索验证方式没有一个高效的可以扩展的系统,提供一个对现有的验证方式进行了效率上的提高,实用性增强的系统。

现有的一些验证方法,多数都只是一些比较抽象的数学和逻辑上的框架,比较缺乏实际可用的系统。少数的代码实现中,基本都是比较直接的计算,没有进行一些系统的设计和一些结构的更改来进行优化。为了实现一个高效的云搜索验证系统,我们主要从事了搜索及验证算法的选择,验证系统的设计,验证系统的优化这几个方面的研究。

\section{论文结构}
本论文主要描述了我们的云搜索正确性快速验证系统的设计与实现,并着重描述其中的整体框架与一些比较重要的细节设计。
余下的部分组织如下。第二章介绍了本课题涉及到的相关工作,主要介绍了RSA Accumulator和一套验证机制。第三章介绍了我们的系统可能的应用场景。第四章描述了我们系统的整体架构以及一些优化方案。第五章我们对本系统做了实验评估,并对实验结果做了简单的分析总结。第六章对本课题进行简要的总结以及对本课题的后续研究做了展望。

\section{本章小结}
本章介绍了我们的课题的研究背景和意义。在目前云计算蓬勃发展的现状下,很多用户选择把数据存储到云服务上,随时随地进行访问查询。由于云计算有潜在的危险性,用户需要的查询结果可能会是错误的,并导致用户利益承受损失。而目前没有一个通用的高效的系统来保证用户搜索结果的正确性。针对这一问题,我们设计并实现了一个较为高效,易于扩展的系统。本章还介绍了对于本课题相关的计算验证方面的目前研究现状。在最后,本章对于本论文的后续章节的内容和结构做了简单的介绍。
