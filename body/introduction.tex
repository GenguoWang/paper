\chapter{绪论}
\label{chap:introduction}


许多因特网和云服务每天产生海量的日志数据,
根据报告~\cite{Logothetis2011}的估算,一个典型数据中心通常托管着上千台机器,
而每台机器每秒要产生1-10MB的半结构化日志数据。
因此,一个中等规模的1000个节点的云平台每天能够产生86TB的日志数据。
这些日志数据不但对于服务提供商非常宝贵,它们可以被用于
预测市场趋势,生成报告,发现欺诈行为~\cite{Oliner2012,ananthanarayanan2013photon},
而且对于系统管理员也非常重要,它们被用于分析故障的根源,
发现系统性能的瓶颈,或者用来检测安全问题~\cite{Fu2009,Logothetis2011,Xu2009}。

因为日志数据异常庞大,而且在持续不断地增长,
减少存储它们的代价成为一个重要的研究课题。
为了降低存储的开销,人们通常先对日志数据进行压缩,这通常能使存储的开销降低到$1/10$。
传统的压缩方法(如\code{gzip})的一个问题是:
它们只有在以一个很大的块为单位进行压缩时才能很好地工作,
而不能独立地解压缩一条单独的日志。
所以即使只是获取一条日志条目,平均下来也需要解压缩半个块的数据。
对于很多日志分析的应用,如日志搜索,
我们需要获取多条不相关的日志条目,这需要从硬盘上读取很多个块并解压缩这些块,
这非常的低效,因为大部分的硬盘带宽和CPU指令都被浪费在并不需要的日志条目上。

如果能够以条目为粒度进行压缩,而非以数据块为粒度,这对于日志分析是至关重要的。
拥有了独立解压缩一条日志项的能力,
我们能够有效地减少从二级存储设备上读取数据和解压缩数据的时间,
从而很好地支持很多日志分析应用场景。
然而, 针对日志数据以条目为粒度进行压缩面临着两个挑战:
首先,很多传统的压缩技术,如自适应模型和算术编码不再适用,
因为这些技术内在本质是一个顺序的风格,它们挖掘利用毗邻的条目间的重复信息。
不能使用这些成熟的技术,我们需要探索一种新的压缩方法。
其次,日志本身的特点给压缩策略带来很多挑战,
每个日志条目通常很短,而且在一条日志间很少有重复的信息;
对于文本,字典只是英文单词,而日志数据有着更大的单词库,如特殊字符,URL等,
从而更难有效地进行编码。
基于这些原因,以条目为粒度的压缩算法很难达到很高的压缩比。

我们提出了一个列向的压缩方法,能够支持独立压缩和解压缩日志条目,
命名为Cowic(\textbf{CO}lumn-\textbf{W}ise \textbf{I}ndependent \textbf{C}ompression)。
我们限制了问题的范围,只针对有着良好格式的日志。
我们方法的基础是半静态的字典模型和哈弗曼编码,
这个组合能够达成独立解压缩这一目标。
由于日志流源源不断有新的数据产生,传统的两遍扫描方法,
即在第一遍扫描时建立模型,第二遍扫描时进行真正的压缩不再使用。
我们的解决方法是使用一小部分的日志做为种子用于训练压缩模型,
然后使用这一模型进行压缩和解压缩,这一方法基于一个假设:
种子的单词分布和整个数据集的分布式相似的。
我们利用了日志数据的结构进行了一系列的改进。
首先,我们将一个日志条目分成几个列,对每一列单独建一个模型,
相同列的数据通常有一些共性,一个单独的模型能够更好地契合这一列而不受其他列数据的干扰。
其次,很多单词经常一起出现,如URL通常以``http://www.''开头,
我们能够自动识别这些模式并将其组织成一个短语,仅用一个代码就能对其进行编码。
然后,我们在训练出的模型之外,构造了一个辅助单词列表,对于种子中不曾出现的单词,
将其添加到列表中,用它的下标编码这个单词。
最后,对于一些常见的有着固定格式的列,我们用一些特殊的模型做进一步的优化,
如时间戳和IP地址等。

我们已经将Cowic实现为一个库,并集成到两个应用中,
一个是日志搜索系统,另一个是日志连接系统。
我们对真实数据的实验评估显示,Cowic在解压缩时间上胜于传统的块压缩算法,
同时有着与之媲美的压缩率。
对于日志搜索系统,当数据在内存时,使用Cowic比\code{gzip}快了3.6-71.1倍;
当数据在硬盘上时,使用Cowic比\code{gzip}快了30.4-246.8\%。
对于日志连接系统,使用Cowic压缩下和不压缩的情况下达到相同的连接质量时,
Cowic压缩只需要占用30\%的内存。

这篇文章主要做了以下贡献:

\begin{itemize}
	\item 我们提出了一个针对具有良好格式的日志的压缩策略,它能够允许
	每一个压缩后的日志条目被独立解压缩。我们已经实现了这一压缩策略,
	并发布为一个C++库~\cite{cowic},开放给公众使用。

	\item 我们将Cowic集成到一个日志搜索系统和日志连接系统中。
	实验结果显示对于日志搜素系统Cowic有更优的搜索时间,
	对于日志连接系统只需更少的内存就能达到相同的连接质量。
\end{itemize}

本篇文论的结构组织如下:
第二章总结了相关工作。
第三章举了两个应用我们压缩策略的场景。
第四章描述了我们压缩方法的设计细节和权衡取舍。
第五章详细列举了实现库的方法。
第六章用真实数据评估了我们的策略。
第七章总结了全文。