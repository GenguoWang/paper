\chapter{应用场景}
\label{chap:application}
在这篇论文中,我们设计并实现了一个云搜索正确性的快速验证系统,该系统能够对于搜索结果进行快速的生成结果证明。有了这个证明,用户可以在本地进行结果正确性的验证。相比于原有的一些验证方案,我们的系统有着高扩展性,高效率的特点。这些特点也让我们的系统有了比较不错的应用场景。在本章中,我们讨论下我们的系统的应用场景。

首先我们的系统需要用户在本地对文档创建索引,以及进行一些必要的计算。所以那些直接在云端的生成新的文件的场景就不太合适使用该系统了。综合考虑下,我们认为本系统的应用场景为提供文档存储和文档全文关键词搜索的云存储系统。

对于一般用户的文档,它一般不会像多媒体文件那么拥有巨大的尺寸,这样方便了我们进行快速的建立索引以及进行一些事先的计算。

下面我们举一个具体的应用场景为例。

\section{云档案管理系统}
一般公司和政府的常规的档案管理都是在企业内部搭建一个服务器,然后部署一套档案管理系统软件\cite{DocumentManage}。在本地部署这样的一套系统有着可以自己定制各种功能,安全性比较高,可以有效避免信息泄露及第三方恶意攻击等问题。但是有些时候,企业或政府部门不需要很复杂的功能,只是需要对于档案的进行增删改查就行,这些功能一般的云服务就可以提供。
而且企业或者政府部分一般有很多报销、税收、政策、审批之类的档案,还有一些档案的扫描件,都是需要大量的存储。这一点云服务也很适合解决。
但是考虑到安全性,比如信息泄露,还有第三方的计算结果正确性无法保证等问题,云服务这种方式就被排除在外了。这个时候,企业或政府部分就只能自己在本地搭建一套档案管理系统。

相比于使用云服务来说,对不是以IT为主要职责企业和政府部门,自己在本地部署一套档案管理系统是需要承受不少额外成本的,而且系统的可靠性和稳定性也无法做到像云服务那么好。


而我们的云搜索正确性快速验证系统可以通过加密来保证档案内容的安全性,可以通过正确性验证来保证用户对档案进行的查询操作的都是正确的。而且我们系统的高效性可以给用户提供友好的体验。我们系统的可扩展性是的它可以容易的应对各种规模的档案数据。

\myfig{app.eps}{4.5in}{云档案管理系统应用场景}{fig:app}{Application for cloud document management}
我们在图\ref{fig:app}描述了将云搜索快速验证系统应用到档案管理的场景。用户在迁移到该系统时,首先要在本地对档案文件建立可验证索引,然后将档案加密再上传到云服务上。之后在日常使用时,用户需要查找某些档案,那么他只需要向云档案管理系统发送查询请求,然后在云端返回结果和结果证明之后,对结果整洁进行验算,就可以保证这次的查询时正确的。

考虑到档案可能不一定都是以文档的形式存在,我们还可以增加对图片这类扫描件的支持。对于一些扫描件图片之类的文件,我们可以使用一些文字识别的技术或者直接是人工标注的方法来建立索引,这样我们的系统也能处理这些非文本的数据了。

\section{本章小结}
本章探讨了下我们系统解决的问题与应用场景。针对我们系统可以给云搜索增加搜索结果验证的功能,我们的应用场景主要瞄准那些对数据安全性,对结果正确性有着高要求的场景。
