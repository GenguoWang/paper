\chapter{全文总结}
\label{chap:conclusion}

本文章针对云搜索正确性验证这一问题,研究了如何建立一个高效实用的云搜索正确性验证系统。
为了解决目前云搜索没有什么合适的验证系统这一问题,我们先是研究了一些目前的验证机制,最后设计了一个基于RSA Accumulator的验证机制作为我们系统的基础。
然后我们根据这个验证机制的特点,设计了一个高效的,具有不错扩展性的系统。
在设计实现这个系统的过程中,我们也遇到了很多问题,比如对于大文档集合搜索的情况,证明生成的速度特别慢。
对此,我们针对如何优化集合元素存在性证明验证这一问题做了调查研究,然后设计并实现了一种树状结构证明来优化正确性证明的生成。

在本章,我们将对本课题的贡献进行总结,并对今后的工作进行展望。

\section{课题贡献}
\begin{enumerate}
\item 
我们设计了一套基于RSA Accumulator的验证机制并在这个验证机制上设计实现了一个云搜索正确性验证系统。
该系统采用了分模块的设计方法,每个模块都是一个单独的进程,
模块之间可以独立部署,使得整个系统有很高的扩展性和易维护性。
实验评估的结果表明,我们的系统相比较于原来的计算方法有着平均为2倍的速度提升,尤其是对于本来处理比较慢的情况,比如原来要10秒左右的查询,我们的系统可以达到4倍左右的提升。
\item
为了解决在搜索的关键词文档集合比较大时,正确性证明生成速度慢的问题,
我们设计的一种树状结构证明,通过将大的集合拆分成多个小的集合并按照树状结构组织起来,来减少计算量,提高并行计算程度,使得证明的生成速度得到了提高。
实验结果表明通过使用树状结构证明,我们的系统在计算正确性证明上有着平均2.5倍的速度提升。
我们还提出了一个将树状结构证明用于完整性证明生成的思路,可以用于提高完整性证明的生成速度。
\item 
为了方便的进行树状结构证明中多个同类任务的并行计算,我们实现了一套轻量级的并行计算框架,该框架可以方便进行任务的分派以及配置使用机器的数量。
通过使用模板编程的方式,该套框架可以方便地扩展到用于计算别的并行任务。
\item
考虑到有些计算效率优先而正确性不用完全保证的情况,我们提出了一种基于采样的验证方式,使得用户可以根据需要在计算效率和正确性保证程度之间做出权衡。

\end{enumerate}

\section{工作展望}
由于能力和时间所限,目前的工作只是在有限时间内完成的力所能及的部分。我们的系统目前总体来看可以用于一般的日常使用,还存在着一些不足的地方和可以进一步提升的地方。比如对于大的数据集的处理速度还比较慢。在今后,我们可以在以下几点继续挖掘:
\begin{enumerate}
\item 实现以树状结构表示的完整性证明。在本文中,我们还只是提出了将树状结构用于完整性证明的方案,却还没有去实现以及进一步的实验验证。从树状结构给正确性证明计算带来的提升来看,用树状结构去计算完整性证明也很可能带来不错的性能提升。
\item 我们目前的系统使用的是基于RSA Accumulator的验证机制,我们可以考虑尝试使用一些别的验证机制\upcite{papamanthou2011optimal},比如可搜索的加密\upcite{curtmola2006searchable,kamara2012dynamic},看看能不能有更好的结构验证性能。
\item 实现更加友好的用户交互界面。目前的系统只提供命令行交互界面,使用起来比较麻烦和复杂,而且对于不懂命令行的用户,更加无从上手。在今后,我们可以实现图形化用户界面和Web界面来提供友好的用户交互界面。
\item 探索更加迅速的数学计算方式。生成证明的过程用到了很多高精度数学计算,这些计算占用了大量的计算时间。如果能够找到提升这些计算速度的方法,那么就可以给整个系统的性能带来很大的提升。我们可以探索下使用GPU进行数学计算\upcite{buatois2007concurrent}这种可能性。
\end{enumerate}
